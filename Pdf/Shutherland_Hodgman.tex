\documentclass{article}
\usepackage[lmargin=1cm, rmargin=1cm, paperwidth=42cm]{geometry}
\usepackage[T1]{fontenc}

\usepackage{MPST}

\newcommand{\tuple}[1]{\langle #1\rangle}

\title{Shutherland-Hodgman~\cite{NeykovaSession2018}}
\date{}
\begin{document}
	\maketitle

\paragraph{Remark.}
Originally, this protocol has one choice where the destination is not the same
across all branches (when choosing $\label{BothIn}$, $\label{BothOut}$ or
$\label{Intersect}$). This may or may not be allowed depending on the specific
formal variant of MPST you choose.

In this benchmark, we reordered them so that the order is always
$\gtComm{P}{R}{{{\ell}{\gtComm{P}{C}{{{\ell}{G}}}}}}$.

	$$
	\gtComm{P}{R}{
		{{Plane}[\tuple{point, point, point, point}]{
				\gtRec{t}{
					\gtComm{P}{R}{
						{{IsAbove}[point]{
								\gtComm{R}{P}{
									{{Res}[bool]{
											\gtComm{P}{R}{
												{{IsAbove}[point]{
														\gtComm{R}{P}{
															{{Res}[bool]{
		\left\{\begin{aligned}
		&\gtComm{P}{R}{
			{{BothIn}{
				\gtComm{P}{C}{
					{{BothIn}[Point]{\gtVar{t}}}
				}
			}}
		}\\
		&\gtComm{P}{R}{
			{{BothOut}{
				\gtComm{P}{C}{
					{{BothOut}[Point]{\gtVar{t}}}
				}
			}}
		}\\
		&\gtComm{P}{R}{
			{{Intersect}[\tuple{Point, Point}]{
					\gtComm{R}{P}{
						{{Res}[point]{
								\gtComm{P}{C}{
									{{SecOut}[point]{\gtVar{t}}}
									{{SecIn}[\tuple{point, point}]{\gtVar{t}}}
								}
						}}
					}
			}}
		}
																	\end{aligned}\right\}
															}}
														}
												}}
											}
									}}
								}
						}}
						{{Close}{
								\gtComm{P}{C}{
								{{Close}{\gtEnd}}
							}
						}}
					}
				}
		}}
	}
	$$

	\bibliographystyle{plain}
	\bibliography{db}
\end{document}
